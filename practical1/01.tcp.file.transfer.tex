\documentclass[a4paper,12pt]{article}
\usepackage[utf8]{inputenc}
\usepackage{graphicx}
\usepackage{listings}
\usepackage{xcolor}
\usepackage{amsmath}

% Thiết lập định dạng cho code
\lstset{
    basicstyle=\ttfamily\footnotesize,
    breaklines=true,
    frame=single,
    numbers=left,
    numberstyle=\tiny,
    keywordstyle=\color{blue},
    stringstyle=\color{red},
    commentstyle=\color{gray},
    showstringspaces=false
}

\title{TCP File Transfer Report}
\author{Your Name \\ ICT Student}
\date{\today}

\begin{document}

\maketitle

\section{Introduction}
The purpose of this project is to implement a simple TCP-based file transfer system using a client-server architecture. The system allows one client to request a file from the server, which sends the requested file back to the client over a TCP connection. The project is implemented in C++ using sockets.

\section{Protocol Design}
The communication between the client and server follows a simple protocol:
\begin{enumerate}
    \item The server waits for a connection from the client.
    \item The client connects to the server and sends the name of the file it wants to download.
    \item The server checks if the file exists:
    \begin{itemize}
        \item If the file exists, the server sends the file's content to the client.
        \item If the file does not exist, the server sends an error message.
    \end{itemize}
    \item The client receives the file (or error message) and saves it locally.
\end{enumerate}

\begin{figure}[h!]
    \centering
    \includegraphics[width=0.8\textwidth]{protocol_design.png}
    \caption{Protocol Design}
    \label{fig:protocol}
\end{figure}

\section{System Organization}
The system is divided into two main components:
\begin{itemize}
    \item \textbf{Server:} A program that listens for client connections, processes file requests, and sends the requested file.
    \item \textbf{Client:} A program that connects to the server, requests a file, and saves the received data.
\end{itemize}

\begin{figure}[h!]
    \centering
    \includegraphics[width=0.8\textwidth]{system_organization.png}
    \caption{System Organization}
    \label{fig:organization}
\end{figure}

\section{Code Implementation}

\subsection{Server Code}
The following code snippet implements the server:
\begin{lstlisting}[language=C++,caption=Server Code]
#include <iostream>
#include <fstream>
#include <cstring>
#include <sys/socket.h>
#include <netinet/in.h>
#include <unistd.h>

#define PORT 5001
#define BUFFER_SIZE 1024

int main() {
    int server_fd, new_socket;
    struct sockaddr_in address;
    int opt = 1;
    int addrlen = sizeof(address);

    // Create socket
    if ((server_fd = socket(AF_INET, SOCK_STREAM, 0)) == 0) {
        perror("Socket failed");
        exit(EXIT_FAILURE);
    }

    // Bind socket to port
    if (setsockopt(server_fd, SOL_SOCKET, SO_REUSEADDR | SO_REUSEPORT, &opt, sizeof(opt))) {
        perror("setsockopt");
        exit(EXIT_FAILURE);
    }

    address.sin_family = AF_INET;
    address.sin_addr.s_addr = INADDR_ANY;
    address.sin_port = htons(PORT);

    if (bind(server_fd, (struct sockaddr *)&address, sizeof(address)) < 0) {
        perror("Bind failed");
        exit(EXIT_FAILURE);
    }

    if (listen(server_fd, 3) < 0) {
        perror("Listen failed");
        exit(EXIT_FAILURE);
    }

    std::cout << "Server is listening on port " << PORT << "...\n";

    if ((new_socket = accept(server_fd, (struct sockaddr *)&address, (socklen_t*)&addrlen)) < 0) {
        perror("Accept failed");
        exit(EXIT_FAILURE);
    }

    char buffer[BUFFER_SIZE] = {0};
    read(new_socket, buffer, BUFFER_SIZE);

    std::ifstream file(buffer, std::ios::binary);
    if (!file) {
        const char* error_message = "File not found";
        send(new_socket, error_message, strlen(error_message), 0);
        std::cerr << "File not found: " << buffer << "\n";
    } else {
        std::cout << "Sending file: " << buffer << "\n";
        char file_buffer[BUFFER_SIZE];
        while (!file.eof()) {
            file.read(file_buffer, BUFFER_SIZE);
            send(new_socket, file_buffer, file.gcount(), 0);
        }
        std::cout << "File transfer completed.\n";
    }

    close(new_socket);
    close(server_fd);
    return 0;
}
\end{lstlisting}

\subsection{Client Code}
The following code snippet implements the client:
\begin{lstlisting}[language=C++,caption=Client Code]
#include <iostream>
#include <fstream>
#include <cstring>
#include <sys/socket.h>
#include <arpa/inet.h>
#include <unistd.h>

#define PORT 5001
#define BUFFER_SIZE 1024

int main() {
    int sock = 0;
    struct sockaddr_in serv_addr;
    char buffer[BUFFER_SIZE] = {0};

    // Create socket
    if ((sock = socket(AF_INET, SOCK_STREAM, 0)) < 0) {
        std::cerr << "Socket creation error\n";
        return -1;
    }

    serv_addr.sin_family = AF_INET;
    serv_addr.sin_port = htons(PORT);

    // Connect to server
    if (inet_pton(AF_INET, "127.0.0.1", &serv_addr.sin_addr) <= 0) {
        std::cerr << "Invalid address / Address not supported\n";
        return -1;
    }

    if (connect(sock, (struct sockaddr *)&serv_addr, sizeof(serv_addr)) < 0) {
        std::cerr << "Connection failed\n";
        return -1;
    }

    std::string filename;
    std::cout << "Enter the filename: ";
    std::cin >> filename;

    send(sock, filename.c_str(), filename.size(), 0);

    std::ofstream output_file("received_" + filename, std::ios::binary);
    if (!output_file) {
        std::cerr << "Error creating file to save data.\n";
        return -1;
    }

    std::cout << "Receiving file...\n";
    int bytes_received;
    while ((bytes_received = read(sock, buffer, BUFFER_SIZE)) > 0) {
        output_file.write(buffer, bytes_received);
    }

    std::cout << "File received and saved as 'received_" << filename << "'\n";

    output_file.close();
    close(sock);
    return 0;
}
\end{lstlisting}

\section{Task Assignment}
\begin{itemize}
    \item \textbf{Member A:} Wrote the server code.
    \item \textbf{Member B:} Wrote the client code.
    \item \textbf{Member C:} Prepared the report and diagrams.
\end{itemize}

\section{Conclusion}
The project successfully implemented a file transfer system over TCP using C++ sockets. The system demonstrates how a client-server model can handle file transfers efficiently.

\end{document}
