\documentclass[a4paper,12pt]{article}
\usepackage[utf8]{inputenc}
\usepackage{graphicx}
\usepackage{listings}
\usepackage{xcolor}
\usepackage{amsmath}

% Định dạng cho mã nguồn
\lstset{
    basicstyle=\ttfamily\footnotesize,
    breaklines=true,
    frame=single,
    numbers=left,
    numberstyle=\tiny,
    keywordstyle=\color{blue},
    stringstyle=\color{red},
    commentstyle=\color{gray},
    showstringspaces=false
}

\title{RPC File Transfer Report}
\author{Your Name \\ ICT Student}
\date{\today}

\begin{document}

\maketitle

\section{Introduction}
This project upgrades the TCP-based file transfer system to use **Remote Procedure Call (RPC)** for better abstraction and ease of use. Using RPC, the client can request a file directly by calling a remote method on the server, and the server handles the transfer seamlessly.

\section{RPC Service Design}
The RPC service implements the following methods:
\begin{itemize}
    \item \texttt{getFile(fileName: string) -> fileContent: bytes}:
    \begin{itemize}
        \item Takes the name of the requested file as input.
        \item Returns the file content if the file exists, or an error message if the file does not exist.
    \end{itemize}
\end{itemize}

\begin{figure}[h!]
    \centering
    \includegraphics[width=0.8\textwidth]{rpc_service_design.png}
    \caption{RPC Service Design}
    \label{fig:rpc_service}
\end{figure}

\section{System Organization}
The system consists of:
\begin{itemize}
    \item \textbf{RPC Server:} Implements the remote method \texttt{getFile}, handles file requests, and sends file data or error messages to the client.
    \item \textbf{RPC Client:} Calls the remote method \texttt{getFile}, receives the file data, and saves it locally.
\end{itemize}

\begin{figure}[h!]
    \centering
    \includegraphics[width=0.8\textwidth]{rpc_system_organization.png}
    \caption{System Organization}
    \label{fig:organization}
\end{figure}

\section{Code Implementation}

\subsection{Server Code}
The server code is implemented as follows:
\begin{lstlisting}[language=C++,caption=RPC Server Code]
#include <iostream>
#include <rpc/rpc.h>
#include <fstream>
#include <cstring>

#define FILE_TRANSFER_PROG ((rpcprog_t)0x31234567)
#define FILE_TRANSFER_VERS ((rpcvers_t)1)
#define GET_FILE_PROC ((rpcproc_t)1)

char* get_file(char* file_name) {
    static char file_content[1024];
    std::ifstream file(file_name, std::ios::binary);

    if (!file) {
        snprintf(file_content, sizeof(file_content), "Error: File '%s' not found", file_name);
    } else {
        file.read(file_content, sizeof(file_content) - 1);
        file_content[file.gcount()] = '\0';
    }

    return file_content;
}

int main() {
    registerrpc(FILE_TRANSFER_PROG, FILE_TRANSFER_VERS, GET_FILE_PROC, 
                (char*(*)(char*))get_file, (xdrproc_t)xdr_wrapstring, (xdrproc_t)xdr_wrapstring);

    std::cout << "RPC Server is running...\n";
    svc_run();
    return 0;
}
\end{lstlisting}

\subsection{Client Code}
The client code is implemented as follows:
\begin{lstlisting}[language=C++,caption=RPC Client Code]
#include <iostream>
#include <rpc/rpc.h>

#define FILE_TRANSFER_PROG ((rpcprog_t)0x31234567)
#define FILE_TRANSFER_VERS ((rpcvers_t)1)
#define GET_FILE_PROC ((rpcproc_t)1)

int main() {
    CLIENT* client;
    char* server_host = "127.0.0.1";
    char file_name[256];
    char* result;

    client = clnt_create(server_host, FILE_TRANSFER_PROG, FILE_TRANSFER_VERS, "udp");
    if (!client) {
        clnt_pcreateerror(server_host);
        return 1;
    }

    std::cout << "Enter the filename: ";
    std::cin >> file_name;

    result = get_file_1(file_name, client);

    if (result) {
        std::cout << "File content:\n" << result << std::endl;
    } else {
        clnt_perror(client, server_host);
    }

    clnt_destroy(client);
    return 0;
}
\end{lstlisting}

\section{Task Assignment}
\begin{itemize}
    \item \textbf{Member A:} Developed the RPC server code.
    \item \textbf{Member B:} Developed the RPC client code.
    \item \textbf{Member C:} Created the report and diagrams.
\end{itemize}

\section{Conclusion}
The project successfully upgraded the TCP file transfer system to an RPC-based system, simplifying the communication process and demonstrating the flexibility of RPC.

\end{document}
